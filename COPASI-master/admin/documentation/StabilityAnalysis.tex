%% LyX 1.3 created this file.  For more info, see http://www.lyx.org/.
%% Do not edit unless you really know what you are doing.
\documentclass[12pt,english]{article}
\usepackage[T1]{fontenc}
\usepackage[latin1]{inputenc}
\usepackage{babel}
\usepackage{graphicx} 


\newcommand{\boldsymbol}[1]{\mbox{\boldmath $#1$}}
\newcommand{\N}{\mbox{$I\!\!N$}}
\newcommand{\Z}{\mbox{$Z\!\!\!Z$}}
\newcommand{\Q}{\mbox{$I\:\!\!\!\!\!Q$}}
\newcommand{\R}{\mbox{$I\!\!R$}}

\makeindex 
\makeatletter
\makeatother

\begin{document}

\title{Stability Analysis in COPASI}

\author{Stefan Hoops\\
 Virginia Bioinformatics Institute \\
 1880 Pratt Dr. \\
 Blacksburg, VA 24060 \\
 USA \\
 shoops@vt.edu$ $}


\date{2004-11-15}

\maketitle
\begin{abstract}
The stability analysis is an important feature of COPASI. Since its
implementation differs from Gepasi's yielding different results an
explanation is in order. This document describes the methods used
within COPASI in detail and explains the differences in comparison
to Gepasi. 
\end{abstract}

\section{Calculation of the Jacobian}

\subsection{Complete Model}
The Jacobian of a vector function $\boldsymbol{f}: \R^{n} \rightarrow \R^{n}$
is defined as:
%
\begin{eqnarray}\label{Jacobian}
\boldsymbol{J}_{i,j}(\boldsymbol{x}) = 
{\left.
  \frac{\partial f_i}{\partial x_j}
\right|}_{\boldsymbol{x}} 
& \mbox{where} & 
i = 1, \ldots, n \;\mbox{and}\; j = 1, \ldots, m 
\end{eqnarray}
%
Let us consider $M$ chemical species participating in $N$
reactions. The kinetics of such a system can then be described by: 
%
\begin{eqnarray}\label{FullModel}
\dot{\boldsymbol{x}} & = & \boldsymbol{N} \, \boldsymbol{v}(\boldsymbol{x})
\end{eqnarray}
%
where $\boldsymbol{N}$ is the stoichiometry matrix, $\boldsymbol{v}$ is the vector
of reaction velocities, and $\dot{\boldsymbol{x}}$ denotes the change
of the chemical species over time.

Using the definition of the Jacobian (\ref{Jacobian}) we derive
for $\boldsymbol{J}$:
%
\begin{eqnarray}
\boldsymbol{J}_{i,j}(\boldsymbol{x}) 
= 
{\left.
  \frac{\partial \dot{x}_i}{\partial x_j}
\right|}_{\boldsymbol{x}} 
= 
\boldsymbol{N}_{i,k} \,  
{\left.
  \frac{\partial v_k}{\partial x_j}
\right|}_{\boldsymbol{x}}
=
\boldsymbol{N}_{i,k} \, \boldsymbol{E}_{k,j}(\boldsymbol{x}) 
\end{eqnarray}
%
where we used the definition of the Elasticity Matrix:
%
\begin{eqnarray}
\boldsymbol{E}_{i,j}(\boldsymbol{x}) = 
{\left.
  \frac{\partial v_i}{\partial x_j}
\right|}_{\boldsymbol{x}} 
\end{eqnarray}

\subsection{Reduced Model}
Let us assume that the Complete Model (\ref{FullModel}) contains $p$
independent species and $q = M - p$ dependent species. Furthermore, we
can assume without loss of generalization that the first $p$ species
are independent. We therefore can describe the dependent species
through the following equations:
%
\begin{eqnarray}\label{MassConservation}
x_i = c_i + \sum_{k=0}^p \alpha_{i,k} \, x_k & \mbox{where} & i = q, \ldots, M,
\end{eqnarray} 
%
which are also know as Mass Conservations or Moieties. 
Using this ability to express the dependent species we rewrite the
velocity function so that the new velocity function $\boldsymbol{\tilde{v}}$
only depends on the first $p$ species $\boldsymbol{x}_r$. 
%
\begin{eqnarray}
\boldsymbol{\tilde{v}}(x_r) = \boldsymbol{v}(x_1, \ldots, x_p, 
c_q + \sum_{i=0}^p \alpha_{q,i} x_i, \ldots, 
c_M + \sum_{i=0}^M \alpha_{M,i} x_i)
\end{eqnarray}
%
This leads to the reduced system:
%
\begin{eqnarray}\label{ReducedModel}
\dot{\boldsymbol{x_r}} & = & \boldsymbol{N}_r \,
\boldsymbol{\tilde{v}}(\boldsymbol{x}_r) 
\end{eqnarray}
%
where $\boldsymbol{N}_r$ is the matrix comprised of first $p$ rows of
$N$.
%
Using the definition of the Jacobian (\ref{Jacobian}) once again we derive
for the Jacobian ($p \mbox{x} p$) of the reduced system:
%
\begin{eqnarray}
\boldsymbol{J}_{i,j}(\boldsymbol{x}_r) 
& = & 
{\left.
  \frac{\partial \dot{x}_i}{\partial x_j}
\right|}_{\boldsymbol{x}_r} 
= 
\boldsymbol{N}_{i,k} \,  
{\left.
  \frac{\partial \tilde{v}_k}{\partial x_j}
\right|}_{\boldsymbol{x}_r} \\
& = &
\boldsymbol{N}_{i,k} \,  
\left(
{\left. \frac{\partial v_k}{\partial x_j} \right|}_{\boldsymbol{x}_r} +
\sum_{l=q}^{M} 
{\left. \frac{\partial v_k}{\partial x_l} \right|}_{\boldsymbol{x}_r}
\frac{\partial x_l}{\partial x_j} 
\right) \\
& = &
\boldsymbol{N}_{i,k} \,  
\left(
\sum_{l=0}^p
{\left. \frac{\partial v_k}{\partial x_l} \right|}_{\boldsymbol{x}_r}
\delta_{l,j}
+ \sum_{l=q}^{M} 
{\left. \frac{\partial v_k}{\partial x_l} \right|}_{\boldsymbol{x}_r}
\alpha_{l,j}
\right) \\
& = & \label{LinkMatrix}
\boldsymbol{N}_{i,k} \, \boldsymbol{E}_{k,l}(\boldsymbol{x}) 
\, \boldsymbol{L}_{l,j}
\end{eqnarray}
%
where $\delta_{l,j} = 0$ for $l \neq j$ and $1$ otherwise. The link
matrix $\boldsymbol{L}$ in (\ref{LinkMatrix}) is defined as: 
\begin{eqnarray}
\boldsymbol{L}_{l,j} & = &
\left\{
\begin{array}{ll} 
\delta_{l,j} & \mbox{if $l \leq p$} \\
\alpha_{l,j} & \mbox{if $l > p$}
\end{array}
\right.
\end{eqnarray}

\subsection{Comparison to Gepasi}
Gepasi's calculation of the Jacobian is based on the reduced
system. However it uses the following slightly different formula
%
\begin{eqnarray}\label{GepasiModel}
\dot{\boldsymbol{x_r}} & = & \boldsymbol{N}_r \,
\boldsymbol{v}(\boldsymbol{x}_r) 
\end{eqnarray}
%
This yields to the Jacobian ($N \mbox{x} p$) of the reduced system: 
%
\begin{eqnarray}
\boldsymbol{J}_{i,j}(\boldsymbol{x}_r) 
& = & 
\boldsymbol{N}_{i,k} \, \boldsymbol{E}_{k,l}(\boldsymbol{x}_r) 
\end{eqnarray}
%
The omission to express the kinetic functions in the new reduced
variables leads to a different Jacobian in the case that the model
contains dependent species.

{\bf Note:} It is important to mention that the method to calculate the Jacobian
is not effecting the calculation of the dynamics of a system in
Gepasi. The reason for this is that Gepasi uses the full
Differential and Algebraic Equation (DAE) system.
 
\section{Stability Analysis}
The stability analysis in both COPASI and Gepasi is based on the
Jacobian of the reduced system since the complete system has a
singular Jacobian whenever it contains dependent species. The
interpretation for the reduced system differs between COPASI and
Gepasi as shown in (\ref{ReducedModel}) and (\ref{GepasiModel}). We
therefore achieve different results for the stability analysis
especially the eigenvalues of the Jacobians differ. In most cases this
does not effect the classification of the steady-state.

We oberved a second cause for different results for the stability
analysis, which is numerical instability. This behviour can be
observed in the Brusellator model ({\tt brusselator.gps}), which is
distributed with Gepasi. \\
\\
COPASI's result:\\
{\tt \small
KINETIC STABILITY ANALYSIS \\
The linear stability analysis based on the eigenvalues \\
of the Jacobian matrix is only valid for steady states. \\
\\
Summary: \\
This steady state is unstable, \\
transient states in its vicinity have oscillatory components \\
\\
Eigenvalue statistics: \\
 Largest real part: 2.5 \\
 Largest absolute imaginary part:  1.65831 \\
 0 are purely real \\
 0 are purely imaginary \\
 2 are complex \\
 0 are equal to zero \\
 2 have positive real part \\
 0 have negative real part \\
 stiffness = 1.00000 \\
 time hierarchy = 0.000000 \\
}
\\
Gepasi's Result: \\
{\tt \small
KINETIC STABILITY ANALYSIS \\
\\
Summary: \\
This steady state is unstable. \\
\\
Eigenvalue statistics: \\
 Largest real part: +1.593070e+000 \\
 Largest absolute imaginary part:  0.000000e+000 \\
 2 are purely real \\
 0 are purely imaginary \\
 0 are complex \\
 0 are equal to zero \\
 2 have positive real part \\
 0 have negative real part \\
 stiffness = 1.72068e+305 \\
 time hierarchy = 1 \\
}
\end{document}
